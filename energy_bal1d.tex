%% SCRATCH NOTES ON ENERGY BALANCE/ FREE ENERGY MINIMIZATION PROBLEMS

We describe the model contained in \begin{verbatim}
energy_bal1d.py\end{verbatim}, as well as the implementation. 

\section{The 1d model} 
	Consider the heat equation on the hemi-sphere. At the 
	inter-tropical convergence zone, the daily heating is most
	intense.  
	
	\begin{defn}{ (solar zenith angle, solar elevation) }
		The \emph{zenith angle} of the sun, $\circ$, i.e., the
		angle made between its rays and a chosen vertical 
		direction
		$e_3$. By definition, solar elevation is its complement.
		
		At \emph{solar noon} 
		$\doteq \argmin_\tau \theta_s(\tau)$,
		zenith angle is minimized, and we have
		\begin{align}
				
		\end{align}
	\end{defn}
	
% Dependencies: \Lip,	
\section{Free energy minimization and its dual}

We state the problem first in a general way, then appeal to its
specific structures. Consider a nonlinear programming problem
in "standard form":
\begin{align}
	\min_{x \in \fancy D} & fe(x)	\label{fe_min}	\\
	\text{subject to }	g_i(x) \leq 0, i \in \{ 1, \ldots, m \}	
						\label{typei}	\\
						h_i(x) = 0, i \in \{ 1, \ldots, p \}
						\label{typeii}
\end{align}
We refer to the relations \eqref{typei} and \eqref{typeii} as
constraints of 'type i' and 'type ii', respectively. Here, $\fancy
D \subset \R^d$ stands for some domain in $\R^d$, $d=2,3$ as yet
unspecified. (Assume $\fancy D$ has non-empty interior.)

Define the Lagrangian function 
$\fancy L:\R^d \times \R^m \times \R^p \rightarrow \R$
\begin{align}
	\fancy L(x,\lambda,\nu) = f(x) 
			+ \sum_{i=1}^m \lambda_i g_i(x) 
			+ \sum_{i=1}^p \nu_i h_i(x).
\end{align}
Here, $(\vec{\lambda},\vec{\nu})$ is the set of \emph{dual
variables} or \emph{Lagrangian multipliers} with
$\vec{\lambda} = (\lambda_i)_i$ etc. 
Consider the mapping 
\begin{align}\label{duality_transform}
	X & : \Lip(\R^{d+m+p} \rightarrow \R) \rightarrow X	\\
	\fancy L(x,\lambda,\nu) \mapsto 
	\inf_{x \in \fancy D} \fancy L(x,\lambda,\nu)
\end{align}
That is, we assume the Lagrangian is Lipschitz.
\begin{prop}
	The function $X(\fancy L)$ is concave in its variables.
\end{prop}

\begin{proof}
	For each fixed $x$ we have
	\begin{align}
		\fancy L(x,\lambda,\nu) = f(x) + \lambda \cdot g(x) 
										+ \nu \cdot h(x).
	\end{align}
	Consider the shadow variable pairs $(\lambda,\nu)$ and
	$(\lambda',\nu')$. Choose $\e > 0$. For each pair, choose
	$x(\e) \in \fancy D$ such that 
	\begin{align}
		L(x(\e),\lambda,\nu) - X[\fancy L](\lambda',\nu')  
		< \frac{\e}{2}
	\end{align}
	and similarly
	\begin{align}
		L(y(\e),\lambda,\nu) - X[\fancy L](\lambda',\nu')
		 < \frac{\e}{2}.
	\end{align}
	Then compute:
	% TODO: finish analysis exer
	\begin{align}
		\eta X[L](\lambda,\nu) + (1-\eta) X(\lambda',\nu') 
		= (something ) + \e.
	\end{align}
	
\end{proof}
 
some geometric/topological definitions.

\begin{defn}
	Let $E$ be a set in $\R^d$. Its \emph{relative interior} is
	defined 
	\begin{align}\label{def:relint}
		\text{relint}(S) \doteq \{ x \in S : \exists \e > 0 s.t. 
									N_\e(x) \cap \text{aff}(S) 
									\subset S \}
	\end{align}
	
\end{defn}

% Dependencies: \aff
\begin{defn}
	The \emph{affine hull} $\aff(S)$ of a set $S \in \R^d$ is
	defined to be the set of all \emph{affine combinations}
	$\sum_{i=1}^K \alpha_i x_i$ of $K$-samples from $S$.
	
	\begin{align}
		\aff(S) = \left\{ \sum_{i=1}^K \alpha_i x_i : k > 0, 
							x_i \in S, \alpha_i \in \R 
							\text{ such that } \sum_{i=1}^K 		
							\alpha_i = 1 \right\}.
	\end{align}

\end{defn}

\begin{defn}
	The \emph{convex hull} of a set $S \in \R^d$ is defined
	to be any of the equivalent
	\begin{enumerate}
		\item The (unique) minimal convex set containing $S$.
		\item The intersection of all convex sets containing $S$.
		\item The set of \emph{convex combinations} of points in
			$X$.
		\item The union of all \emph{simplices} with vertices in
		$X$.
	\end{enumerate}
\end{defn}

% Intuition: rubber band
\begin{prop}
	For bounded sets in $\R^2$, the boundary of the convex hull
	is the simpled closed curve with minimum perimeter containing
	$S$.
\end{prop}

The above does not "immediately" generalize to higher dimensions.
A neighborhood of 




\begin{remark}
	Q: How do these definitions compare? 
\end{remark}

















	
	
	
	

