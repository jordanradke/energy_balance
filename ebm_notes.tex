ENERGY BALANCE MODEL

 (incoming radiation from sun) ~ (outgoing energy from Earth)

 Q1: Can temperature be accurately predicted/explained over
   historical timescales? 
 Q2: What do the steady-state solutions of a 1d (2d) energy-balance
       model look like?
 A2: HW3 below.
 Q3: Can the bifurcation of steady-state solutions be quantitatively
     described? See Cardy on renormalization and order parameters.
     Also see Chen-Hale/Verhulst(Nayfeh?) on bifurcations,  

 Implementation: The Boltzmann equation evolves the densities
   \begin{enumerate}
       \item Mass: $\rho_a(x_\alpha,t_m) = a, \ldots, K$ where
                $(\alpha,m)$ denotes a space-time grid point. Let
                $f_{\alpha\beta m}^a \doteq \frac{1}{|Q_{\alpha}||\tau_m|}\int_{Q_\alpha \times \tau_m} f_0^a dx dv dt$ 
                By testing the equation
                \begin{align}\label{boltzmann_vlasov}
                   \D_t F^a + v \cdot \nabla_x F^a - \nabla_x \Phi^a_\gamma \cdot \nabla_v F^a = Q(F^a, \sum_{b=1}^K F^b)
                \end{align}
                The left-hand side represents the transport of the densities $F^a$; in its linearized form it has an
                explicit representation formula in terms of the backwards trajectories; see [Vidav70s], "Nuetron
                Transport". By using this "double-duhamel" representation of the solution and the detailed 
                decomposition of the linearized operator $L^a[f] = -\nu^a(x,v) \vec{f}(t,x,v) + K^a[\vec{f}]$ due to
                Grad [40s-50s] (but see [Glassey96] for more recent treatment), we wish to prove estimates of the form:
                \begin{align}\label{stability_est}
                   \sup_{t\leq T_0} \norm{ F(t) - \mu }_{\L^p_w(\Omega \times \R^3)} 
                   \leq C^1_{d,p,\Omega,H,T_0}\norm{ F_0 }_{\L^p_w(\Omega \times \R^3)}
                       + C^2_{d,p,\Omega,H,T_0}\fancy H[F_0]
                \end{align}
                
                   
                In particular, guided by the Lyupanov functional and the weighted $L^\infty$ estimates that we expect to
                hold for the nonlinear term look for solutions
                \begin{align}\label{weighted_Lp_spaces}
                   \L^p_{w_{H,\beta,\beta_1}}(\Omega \times \R^3) 
                   =  \left( \int w_{H, \beta, beta_1} |f^a(t,x,v)|^p dx dv \right)^{1/p}
                \end{align}
                \begin{align}\label{conservation_laws}
                   \fancy H[f] \doteq
                \end{align}
                Q: in the single-species or multi-species case, having $-\int F^a \log F^a$ low => ?
 Assume further a "moderate diurnal cycle" due to
   \begin{enumerate}
       \item Large heat capacity of earth:
       \item Fast rotation of earth:
   \end{enumerate}
 Note: there is also diurnal convective action transporting
 water vapor and, one assumes, CO2 (passive tracer). 

 At the scales at which particles interact with each other,
 intermolecular forces may be the primary organizing factor
 in the steady states (i.e. thermodynamic equilibria)
 you expect to see for a given system of fixed volume, particle
 numbers $N_1, \ldots, N_k$, (and so $\sum_k N_k \doteq N$ their
 total is also conserved), and bulk-conserved mass, momentum and
 energy.

 Task 1: Write down Hamiltonian of the interacting particle system:
 H(t,X,V) = \frac{1}{N} \sum_{i=1}^N |V_i|^N + \frac{\alpha}{N^2} \sum_{i,j=1}^N \phi_{ij}^{\gamma_{ij}}(X_i-X_j) 
 where
 
 Q1: How is this energy conservation a restatement of the 1st law of thermo (energy balance)?
 Dependencies:
import numpy as np
import scipy
import matplotlib as plot
from shapely.geometry.polygon import Polygon
from itertools import permutations

    L = 10 # [km]
    
    # construct ordered sequence of point tuples: the 'shell' 
    # corners
    hull = [ [L,L],[-L,L], [-L,-L],[L,-L] ]
    omega = Polygon(hull)
    
    # test 1:
    p = plot.plot(*omega.exterior.xy)
    plot.show()
    

def vol(omega)
    return 
